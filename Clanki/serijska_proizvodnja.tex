\subsection{Serijska proizvodnja}
Če so vse mere urejene, se lahko začne serijska proizvodnja. Ko vstavimo novo palico, skoraj vedno prvi izdelan kos ni dober, ker imajo palice na obeh koncih pobran rob in neravno čelno ploskev. To bi se dalo rešit, če bi v ciklu izveli še operacijo čeljenja, ampak pri velikih količinah, se bolj splača prvi kos zavrečt, kot pa dodajanje nove operacije.
Ko stroj obdela celotno palico, ima podajalec stikalo, ki se sproži, ko povleče suport zadnji kos. To nenudoma vstavi celoten stroj in prižge signalno luč, ki nam da vizualno signalizacijo, ker je v delavnici polnih stružnih avtomatov največkrat zelo glasno. 