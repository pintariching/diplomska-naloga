Orodja na avtomatih so največkrat narejena iz HSS,
hitroreznega jekla. To so jekla z dodanim kromom, volframom,
molibdenijem, vanadijem in kobaltom. Na splošno, skupna količina
legirnih elementov ne presega 7\%. Trdnost teh jekel je zelo visoka,
od 63 do 67 HRC.

Uporabljajo pa se prav zaradi te visoke trdnosti. Z uporabo
brusilnega stroja ali brusilke, se surovec obdela v nož po želji,
moramo pa paziti, da pri brušenju jekla ne pregrejemo, saj tako
hitro zgubi svojo trdoto.
Iz njih lahko naredimo katerokoli orodje; stružne, odrezne,
profilne nože, kot tudi topovske svedre in pestiče. Surovci so
največkrat dolgi 200 mm in kvadratni z standardnimi merami
(8x8, 10x10, 12x12, 16x16…). Na spodnji sliki \ref{hss_nozi}
so prikazani HSS surovci, kateri navadno že imajo brušene
nekatere površine, za lažje nadaljno brušenje.

\begin{figure}[H]
	\begin{center}
		\includegraphics[width=10cm]{hss_nozi.jpg}
		\caption{HSS surovci
			\cite{hss_nozi}}
		\label{hss_nozi}
	\end{center}
\end{figure}

Obstojnost hitroreznih jekel v primerjavi z ostalimi ni najboljša,
lahko se pa izboljša z raznimi prevlekami. Zaradi tega, se za
avtomatna jekla uporablja večinoma HSS, za trša jekla, nerjavna
in poboljšana, se pa uporabljajo noži iz karbidnih trdnin,
narejenih z postopkom sintranja ali WIDIA, ki je pa zelo trda
kovinska zlitina, katere ime je izpeljano iz besed "Wie Diamant",
kar po slovensko pomeni \textit{kot diamant}.

\subsubsection{Rezalna hitrost}
Rezalna hitrost je odvisna od materiala, ki ga obdelujemo,
obdelovalnega postopka, prereza odrezka, hitrosti podajanja
in željene površine (groba ali fina). Vrednosti rezalne hitrosti
najdemo v tabelah, ki so bile narejene na podlagi preizkušanja
in je nikoli ne računamo. Spodaj na sliki \ref{rezalna_hitrost}
je primer tabele rezalnih hitrosti vzetih iz Krautovega strojniškega
priročnika.

\begin{figure}[H]
	\begin{center}
		\includegraphics[width=10cm]{rezalne_hitrosti.jpg}
		\caption{Tabela rezalnih hitrosti
			\cite{strojniski_prirocnik}}
		\label{rezalna_hitrost}
	\end{center}
\end{figure}

To rezalno hitrost \(V_c\) nato uporabimo v enačbi
\begin{equation}
	\begin{split}
		V_c &= \frac{\pi*D*n}{1000},
	\end{split}
\end{equation}
kjer je \(D\) premer obdelave in lahko izpostavimo obrate glavnega
vretena \(n\) v \(\frac{obr}{min}\) kot
\begin{equation}
	\begin{split}
		n &= V_c * \pi * D * 1000,
	\end{split}
\end{equation}
nato lahko izračunamo hitrost pomika orodja \(f\) v \(\frac{mm}{min}\)
\begin{equation}
	\label{pomik_orodja}
	\begin{split}
		f &= n * f_z * Z,
	\end{split}
\end{equation}
kjer je:
\begin{itemize}
	\item[--] \(f_z\) -- pomik orodja na en zob in
	\item[--] \(Z\) -- število zob.
\end{itemize}
Pomik na zob \(f_z\) izračunamo z
\begin{equation}
	\label{pomik_na_zob}
	\begin{split}
		f_z &= \frac{f}{n * Z}.
	\end{split}
\end{equation}
Vidimo, da sta \eqref{pomik_orodja} in \eqref{pomik_na_zob} druga od
druge odvisni, zato največkrat sami izberemo \(f_z\) glede na to,
kako togo orodje in stroj imamo, ter kakšen material obdelujemo.
Za mehkejše in lažje obdelovalne materiale izberemo večji \(f_z\)
za trdnejše pa manjšega. Pomik na zob izbiramo tudi glede na vrsto obdelave,
če gre za fino obdelavo, kjer so zahtevnejše tolerance in je zahtevana
hrapavost površine, vzememo primerno manjše vrednosti.

\subsubsection{Vrtanje}
Je operacija obdelave materiala z odrezovanjem.
Pri vrtanju se lahko vrti orodje, obdelovanec ali oba hkrati,
pri čemer za izračun pomika, vrtljaje seštevamo, če se te vrtijo
v nasprotno smer. Rezultat operacije je izvrtina valjaste oblike.
Vrtamo z svedri, ki imajo za vsak material določen rezni kot in kot vzpona
vijačnice. Npr. za medenino, je kot vzpona zelo velik, da se
krhki ostružki hitro odstranijo iz luknje. Za jekla, kjer so
ostružki največkrat dolgi in nepretrgani, uporabljamo svedre z
zelo majhnim kotom vzpona.

Svedri za kovino so večinoma trdokovinski ali pa narejeni iz
hitroreznega jekla in prevlečeni z titanovo zlitino.
Naziv »spiralni sveder« ki se ponekod še uporablja je napačen.
Spirala je krivulja na ravnini, medtem ko je vijačnica krivulja
skozi prostor, okoli osi, zato bi praviloma morali uporabljati
izraz vijačni sveder.
Poznamo še lasersko vrtanje, ki se uveljavlja v zahtevnih
proizvodnjah, kjer je potrebno izdelati zelo majhne luknje,
60 - 150 µm, kar z svedri ni mogoče. Lasersko vrtanje pa je zelo
uporabno tudi pri vrtanju poševnih izvrtin, kar je za običajno
vrtanje velik izziv.
Obstaja pa še elektro-erozijske prebijalke, ki z bakrenimi ali
grafitnimi elektrodami lahko prebijejo le elektro prevodne materiale.

Za primer imamo spodaj na sliki \ref{hss_sveder} primer navadnega
vijačnega HSS svedra prevlečenega z titanovo zlitino.
\begin{figure}[H]
	\begin{center}
		\includegraphics[width=8cm]{sveder.jpg}
		\caption{Navaden HSS sveder
			\cite{sveder}}
		\label{hss_sveder}
	\end{center}
\end{figure}
\subsubsection{Povrtavanje}
Je operacija vrtanja, katere namen je izboljšanje točnosti in
hrapavosti površine v že narejeni izvrtini. Največkrat se
uporablja ko imamo ujem in moramo doseči gladko površino v
mejah natančnosti IT6 do IT11. Orodja za povrtavanje, povrtala,
imajo 6 ali več rezil. Na spodnji sliki \ref{povrtalo} lahko vidimo
primer ročnega povrtala z šestimi rezili, ki se uporablja na
univerzalni stružnici.
\begin{figure}[H]
	\begin{center}
		\includegraphics[width=10cm]{povrtalo.jpg}
		\caption{Povrtalo
			\cite{sts_arhiv}}
		\label{povrtalo}
	\end{center}
\end{figure}

\subsubsection{Vrezovanje navoja}
Posebna oblika povrtavanja je vrezovanje navoja. Ta operacija se
lahko izvede samo takrat, ko imamo že predvrtano pravilno
velikost luknje za dani navoj. Orodja za vrezovanje navojev se
imenujejo navojni svedri, prikazani na spodnji sliki \ref{navojni_sveder}.
\begin{figure}[H]
	\begin{center}
		\includegraphics[width=8cm]{navojni.jpg}
		\caption{Navojni sveder
			\cite{sts_arhiv}}
		\label{navojni_sveder}
	\end{center}
\end{figure}
Ta operacija je še posebej zahtevna pri katerikoli vrsti obdelave,
saj je za vrezovanje navoja potreben velik navor, kot tudi zelo natančna
sinhronizacija med glavnim in podajalnim gibanjem. Pri struženju
na univerzalni stružnici to največkrat ni problem, saj lahko navojni
sveder vpne v konjička, ki se lahko prosto pomika po vodilih.
Pri izdelavi navoja na CNC strojih pa moramo paziti, da pravilno vpišemo
obrate in pomike, ali pa uporabimo kompenzacijsko glavo, v katero se
vpne navojni sveder in nam ta glava omogoča premikanje svedra, če
ne moremo točno uskladiti obratov in pomikov, npr. Pri starejših
strojih.

Na stružnih avtomatih, kjer so obrati glavnega vretena navadno konstantni
in tudi navadno nemoremo obrniti smeri vrtenja, uporabimo trik, kjer
spreminjamo hitrost vrtenja navojnega svedra. Če imamo 2000 \(\frac{obr}{min}\)
na glavnem vretenu in 2200 \(\frac{obr}{min}\) v nasprotnem vretenu, kjer je
vpet navojni sveder, se te obrati odštejejo in lahko smatramo, kot da vrezujemo
navoj samo z 200 \(\frac{obr}{min}\). Ko pa hočemo navojni sveder izvleči, pa samo
zmanjšamo hitrost na nasprotnemu vretenu na npr. 1600 \(\frac{obr}{min}\)
in ker se to vrti počasnejše od glavnega, lahko sveder izvlečemo iz navoja.

\subsubsection{Grezenje}
Je postopek širjenja že obstoječe. Orodja imenujemo grezila,
ki vrtajo kvalitetnejšo luknjo kot sveder ali pa pripravlja
luknjo za povrtavanje. Z grezili dosegamo natančnosti luknje od
IT8 do IT10. Poznamo tri vrste grezenja: Grobo, fino in oblikovno grezenje.

Za grobo grezenje se največkrat uporablja samo navadni vijačni
sveder, za fino grezenje uporabljmao grezila, za oblikovno grezenje
pa uporabljamo razna stožčasta enorezilna ali večrezilna grezila
za oblikovanje izvrtin.

Spodaj na sliki \ref{grezilo} je prikazan primer grezenja
luknje za vijak z konusno glavo.

\begin{figure}[H]
	\begin{center}
		\includegraphics[width=4cm]{grezenje.jpg}
		\caption{Primer grezenja luknje za skritje vijaka
			\cite{sts_arhiv_grezenje}}
		\label{grezilo}
	\end{center}
\end{figure}

\subsubsection{Struženje}
Struženje je najbolj razširjen postopek odrezavanja za \
obdelavo valjastih obdelovancev, možno pa je stružiti tudi ravne ploskve
in celo nekatere neokrogle oblike, če orodje med delom niha sinhronizirano
z vrtenjem obdelovanca.

Glavno gibanje pri struženju je rotacijsko in ga opravlja
obdelovanec. Podajalno gibanje je navadno premočrtno ali pa
sledi poljubni krivulji. Struženje zavzema velik delež celotne
obdelave z odrezavanjem, zato je struženje najnatančneje
raziskan postopek. Raziskovanja pri struženju so dala vrsto
zakonitosti, ki so osnove za vse postopke odrezavanje kovin.

Poznamo vzdolžno - slika \ref{vzdolzno_struzenje},
prečno - slika \ref{precno_struzenje}, stožčasto, profilno,
zarezno struženje - slika \ref{struzenje_utora}, kot tudi struženje navoja, podstruževanje in kopirno struženje.

\begin{multicols}{3}
	\begin{figure}[H]
		\includegraphics[width=\linewidth]{struzenje_utora.jpg}
		\caption{Struženje utora
			\cite{sts_arhiv_struzenje}}
		\label{struzenje_utora}
	\end{figure}

	\columnbreak

	\begin{figure}[H]
		\includegraphics[width=\linewidth]{vzdolzno_struzenje.jpg}
		\caption{Vzdolžno struženje
			\cite{sts_arhiv_struzenje}}
		\label{vzdolzno_struzenje}
	\end{figure}

	\columnbreak

	\begin{figure}[H]
		\includegraphics[width=\linewidth]{precno_struzenje.jpg}
		\caption{Prečno struženje
			\cite{sts_arhiv_struzenje}}
		\label{precno_struzenje}
	\end{figure}
\end{multicols}
