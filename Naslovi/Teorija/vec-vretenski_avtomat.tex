So izboljšani tip eno-vretenskih avtomatov. Imajo od
2 do 8 vreten ampak se večinoma uporabljajo 4 ali 6
vretenski avtomati. Cikel obdelovanja kosa se zaključi,
ko se revolver z obdelovanci obrne za 1 obrat. Na vsaki stopnji
obrata se izvede ena stopnja obdelave, zato je skupni čas za en
kos enak eno-vretenskemu avtomatu, le produktivnost je veliko
večja. Čas ene stopnje obdelave je odvisna od časa najdaljše
obdelave na kosu. Npr. če imamo 5 postaj na katerih se struži 5s
in eno postajo na kateri se vrta 10s, se bo vreteno obrnilo vsakih 10s.

Na spodnji sliki \ref{vec_vretenc} je prikazana shema gibov več-vretenskega
avtomata.

\begin{figure}[H]
	\begin{center}
		\includegraphics[width=\linewidth]{vec_vretenski_avtomat_shema.jpg}
		\caption{Shema več vretenskega avtomata
			\cite{vec_vretenska_struznica_shema}}
		\label{vec_vretenc}
	\end{center}
\end{figure}