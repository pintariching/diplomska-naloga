\newpage
\section{Zaključek}
Celotna nastavitev lahko traja tudi več delovnih dni in zahteva
izobrazbo, ki v današnjih časih počasi izumira. Krivuljne avtomate
počasi nadomeščajo hitrejši in bolj natančni CNC-stroji, ki so kljub
napredni tehnologiji velikokrat mnogo počasnejši od krivuljnih strojev, saj
imajo ti možnost kombinirati več obdelav v istem času. Še ena prednost avtomatov je, da so veliko cenejši
za vzdrževanje in delovanje. Vzrok napake se hitro najde in vse v njih je
mehansko. Opazil sem tudi stroj Wolf TSM 280 podjetja Wolf Machinenbau AG,
ki bi lahko bil konkurenčen s krivuljnim avtomatom. Možnost vpeljave
takšnega stroja namesto več krivulnih avtomatov bi bilo potrebno
še bolj raziskati. Bolj podrobno bi se lahko tudi opisal
sam izračun za kompleksnejše gibe, kot so radiusi, in še posebej
ročno risanje takšnih krivulj s šablonami.