\newpage
\textbf{\fontsize{14}{21}\selectfont ZAHVALA} \\
Najprej bi se zahvalil svojemu očetu, Viliju Pintariču,
ki se je odločil nadaljevati v korakih svojih staršev in 
je ustanovil podjetje AVIKO d.o.o. in mi z tem omogočil delo
v domačem okolju.

Zahvalil bi se tudi svoji mami, Mojci Pintarič, ki skupaj z očetom
dela na družinskem podjetju. Naučil sem se marsikaj stvari o 
logistiki nabave, prodaje ter tudi nekaj stvari o upravljanju
financ podjetja.

Zahvala gre tudi celotnemu kolektivu podjetja AVIKO, za vso
raznovrstno novo pridobljeno znanje in za zelo prijazno in velikokrat
zabavno delovno okolje.

\newpage
\textbf{\fontsize{14}{21}\selectfont POVZETEK} \\
V tej seminarski nalogi bom v uvodu predstavil podjetje AVIKO d.o.o.,
njihove izdelke in kapacitete izdelave, ter tudi nekaj malega
o zgodovini podjetja. Bolj v podrobnosti bom pa opisal
postopek nastavitve stružnega avtomata ter nekaj problemov in rešitev,
ki so nastali ob tem. Na koncu tega projekta je bil moj cilj
nastaviti stroj tako, da znotraj zahtevane tolerance izdela 
izdelek v najkrajšem možnem času.

\textbf{\fontsize{14}{21}\selectfont Ključne besede:} \\
\fontsize{12}{16}AVIKO d.o.o., stružni avtomat, krivulja, struženje

\newpage
\textbf{\fontsize{14}{21}\selectfont ABSTRACT} \\
In the introduction of this seminar work, I will present the 
company AVIKO d.o.o, their products, production capacities
and a little bit about their history. I'll dive deeper into the set up of 
cam lathes and list some problems and solutions that I discovered
while tackling this project. My goal of this project was to set up 
a cam lathe so, that it produces parts within a given tolerance 
in the least amount of time.

\textbf{\fontsize{14}{21}\selectfont Key words:} \\
AVIKO d.o.o., cam lathe, cam, turning