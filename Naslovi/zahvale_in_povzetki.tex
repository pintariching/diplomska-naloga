\newpage
\textbf{\fontsize{14}{21}\selectfont ZAHVALA} \\
Najprej bi se zahvalil svojemu očetu, Viljemu Pintariču,
ki se je odločil nadaljevati v korakih svojih staršev in
je ustanovil podjetje AVIKO d.o.o. in mi s tem omogočil delo
v domačem okolju.

Zahvalil bi se tudi svoji mami, Mojci Pintarič, ki skupaj z očetom
dela v družinskem podjetju. Naučil sem se veliko stvari o
logistiki nabave, prodaje ter tudi nekaj stvari o upravljanju
financ podjetja.

Zahvaljujem se tudi svojemu staremu očetu, Viljemu Pintariču,
ki me je naučil dela s stružnico in me vpeljal v to področje strojništva.

Zahvala gre tudi celotnemu kolektivu podjetja AVIKO za vso
raznovrstno novo pridobljeno znanje in za zelo prijazno ter velikokrat
zabavno delovno okolje.

\newpage
\textbf{\fontsize{14}{21}\selectfont POVZETEK} \\
\\
V podjetju AVIKO d. o. o. se za masovno izdelavo enostavnejših
izdelkov uporablja krivuljne stružne avtomate. To so starejši
stroji, ki za krmiljenje uporabljajo krivulje in vodila namesto
sodobnih servo ali koračnih motorjev. Takšni stroji so veliko
hitrejši od sodobnih CNC-strojev, njihova slabost pa je, da niso
tako natančni, nastavitev stroja pa je precej počasnejša in zahtevnejša.
V tej nalogi bom v uvodu predstavil podjetje AVIKO d.o.o.,
za kaj je specializirano, kako se je razvijalo skozi
čas ter kam je usmerjeno. Predstavil bom tudi njihov strojni
park in postopek, kako iz surovega paličastega materiala
izdelajo razne kompleksne izdelke.
Bolj podrobno bom opisal postopek struženja, delovanja
stružnega avtomata, postopek
izračuna in izdelave krivulj za krmiljenje stružnega avtomata ter na kratko opisal, kako se dela s takšnimi stroji.
Moj cilj na koncu tega projekta je bil
nastaviti stroj tako, da izdela
izdelek v najkrajšem možnem času, in preveriti, ali lahko stroj
dosega zahtevane tolerance. Predstavil bom tudi, kako se
izdelek nadaljno še obdela in pregleda, preden se ga dostavi kupcu.

\textbf{\fontsize{14}{21}\selectfont Ključne besede:}
\fontsize{12}{16}stružni avtomat, krivulja, struženje, proizvodnja

\newpage
\textbf{\fontsize{14}{21}\selectfont ABSTRACT} \\
\\
The company AVIKO d. o. o. uses swiss cam lathes to mass produce
simple parts. These old machines use a system of cams and followers
for controll instead of modern servo or stepper motors. They are
much faster than modern CNC machines, but their downside is,
that they are less precise and the set up is a lot slower and demanding.
In the introduction of this thesis, I'll present the company
AVIKO d. o. o., their specializations and how the company developed
through time and where it's headed. I'll also present their
machinery park and the process of how they turn bar stock into
various complex products. Then I'll describe more in depth the
process of turning, the workings of a swiss cam lathe, how to
calculate and make cams that controll the lathe and how to work
with them. My main goal at the end if this thesis was, to set up
a swiss cam lathe, so that it produces a part in the least amount
of time and check if the machine can achieve the required
tolerances. I'll also describe how the parts are further processed
and inspected before delivered to the customer.

\textbf{\fontsize{14}{21}\selectfont Key words:}
\fontsize{12}{16}swiss cam lathe, cam, turning, production