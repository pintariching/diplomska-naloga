Za serijsko proizvodnjo enostavnejših izdelkov se povsod po svetu veliko
uporabljajo krivuljni stružni avtomati. Takšni starejši stroji so
v nasprotju z sodobnejšimi CNC stroji veliko enostavnejši v delovanju,
a hkrati zahtevajo več znanja za nastavitev. Delujejo na preprostem
principu krivulje in sledilca, ki preko vzvodov krmili delovanje.
Problem tiči v izračunu in izdelavi takšnih sistemov, saj se najmanjše
nepravilnosti ali napake poznajo na končnem izdelku. Potrebno je
veliko znanja iz področja odrezovanja kovin, toplotne obdelave
in strojnih elementov, za določanje pravilnih rezalnih hitrosti,
natančne izdelave in toplotne obdelave krivulj ter za pravilno
montažo in sestavitev orodij in stroja. Veliko tega znanja se je
že izgubilo in ga je v sodobnih časih potrebno na novo odkrivati,
saj podjetniki velikokrat nočejo tega deliti svoji konkurenci.
Namen te diplomske naloge je, da predstavi manjši del tega znanja
z primerom postopka nastavitve krivuljnega avtomata.