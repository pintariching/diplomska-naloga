AVIKO d.o.o. je družinsko proizvodno podjetje,
ki se ukvarja s serijsko mehansko obdelavo in sicer
s CNC tehnologijo struženja in s struženjem na
konvencionalnih stružnih avtomatih.
Podjetje ima na tem področju več kot 20 let izkušenj,
ki jim s pomočjo sodobnega strojnega parka omogočajo
izdelavo najzahtevnejših proizvodov.

Visok tehnološki nivo podjetja dokazuje njihov strojni
park in avtomatiziran sistem za pregledovanje in
kontrolo končnih izdelkov s kamero, s čimer lahko
zagotovijo izpolnjevanje strogih zahtev kakovosti.

\subsubsection{Izdelki}
Podjetje AVIKO d.o.o. je specializirano podjetje za izdelavo manjših
kosov na CNC stružnicah v večjih serijah. Za uporabljajo dve
vrste strojev - specializirane CNC dolgostružne stružnice za preciznost
in kompleksne obdelave, ter krivuljne avtomate, ki so specializirani za
hitro in poceni obdelavo. Podjetje se največ ukvarja z izdelavo izdelkov
za avtomobilsko industrijo - razne puše, podložke, po meri vijaki in podobno.

\subsubsection{Kapacitete}
Kapacitete so odvisne od zahtevnosti kvalitete, količine
in od kompleksnosti samega izdelka. Če je izdelek enostaven
lahko podjetje izdela tudi čez nekaj miljonov izdelkov letno,
tudi ob strogih zahtevah kvalitete izdelkov, če so le-te enostavnejši
za izdelavo.

\subsubsection{Zgodovina}
Zgodovina podjetja sega vse do leta 1985, ko je Viljem Pintarič (starejši)
začel z proizvodnjo struženih kovinskih komponent v svoji garažni delavnici.
Kasneje se mu pridruži še njegova žena Cvetka Pintarič in ustanovita
podjetje Kovinostrugarstvo Cvetka Pintarič d.o.o.

Leta 2005 je ob upokojitvi svojih staršev, Viljem Pintarič (mlajši)
prevzel podjetje ustanovil AVIKO d.o.o.

Leta 2008 se proizvodnja preseli iz garažne delavnice pod našo hišo
v Dobličih na zdajšno lokacijo v industrijski coni TRIS kanižarica,
kjer novi proizvodni prostori merijo približno 600 m²

Leta 2010 je podjetje začelo dobivati naročila iz tujine,
prvo iz Nemčije, kasneje iz Avstrije in Slovaške.

V 2011 se tehnološko posodobi proizvodnja
in se nabavi prvi CNC stroj.

Leta 2012 razširijo ponudbo storitev z zagonom nove linije za praškasto barvanje
kovin in ponudbo storitev na novi žični eroziji.

Leta 2015 pa se podjetje reorganizira po standardu ISO 9001.

Leta 2021 podjetje dobi optični merilni stroj ViciVision za boljše spremljanje
kvalitete proizvodnje.

\subsubsection{Cilji}
Podjetje se kljub svojem omejenem prostoru stalno razvija in izboljšuje
svoje postopke. Trenutno ima na voljo štiri robote za samodejno pregledovanje
in merjenje izdelkov z merilnimi uricami in optično merjenje z kamerami.
V prihodnosti imajo cilj razširiti strojni park z večimi krivuljnimi avtomati
in CNC stroji.

\newpage