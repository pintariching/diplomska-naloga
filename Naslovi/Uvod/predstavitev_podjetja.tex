AVIKO d. o. o. je družinsko proizvodno podjetje,
ki se ukvarja s serijsko mehansko obdelavo, in sicer
s CNC-tehnologijo struženja in s struženjem na
konvencionalnih stružnih avtomatih.
Podjetje ima na tem področju več kot 20 let izkušenj,
ki jim s pomočjo sodobnega strojnega parka omogočajo
izdelavo najzahtevnejših proizvodov.

Visoktehnološki nivo podjetja dokazuje njihov strojni
park in avtomatiziran sistem za pregledovanje ter
kontrolo končnih izdelkov s kamero, s čimer lahko
zagotovijo izpolnjevanje strogih zahtev kakovosti.

\subsubsection{Izdelki}
Podjetje AVIKO d. o. o. je specializirano podjetje za izdelavo manjših
kosov na CNC-stružnicah v večjih serijah. Za to uporabljajo dve
vrsti strojev -- specializirane dolgostružne CNC-stružnice za preciznost
in kompleksne obdelave ter krivuljne avtomate, ki so specializirani za
hitro in poceni obdelavo. Podjetje se največ ukvarja z izdelavo izdelkov
za avtomobilsko industrijo -- raznih pušev, podložkov, vijakov po meri in podobno.

\subsubsection{Kapacitete}
Kapacitete so odvisne od zahtevnosti kvalitete, količine
in od kompleksnosti samega izdelka. Če je izdelek enostaven,
lahko podjetje izdela tudi čez nekaj miljonov izdelkov letno
tudi ob strogih zahtevah kvalitete izdelkov, če so le-ti enostavnejši
za izdelavo.

\subsubsection{Zgodovina}
Zgodovina podjetja sega vse do leta 1985, ko je Viljem Pintarič (starejši) v svoji garažni delavnici začel s proizvodnjo struženih kovinskih komponent.
Kasneje se mu je pridružila še njegova žena Cvetka Pintarič in ustanovila sta
podjetje Kovinostrugarstvo Cvetka Pintarič d. o. o.

Leta 2005 je ob upokojitvi svojih staršev Viljem Pintarič (mlajši)
prevzel podjetje in ustanovil AVIKO d. o. o.

Leta 2008 se je proizvodnja preselila iz garažne delavnice pod našo hišo
v Dobličih na zdajšno lokacijo v industrijski coni TRIS Kanižarica,
kjer novi proizvodni prostori merijo približno 600 m².

Leta 2010 je podjetje začelo dobivati naročila iz tujine,
najprej iz Nemčije, kasneje iz Avstrije in Slovaške.

V letu 2011 se je proizvodnja tehnološko posodobila 
in prišlo je do nakupa prvega CNC-stroja.

Leta 2012 so razširili ponudbo storitev z zagonom nove linije za praškasto barvanje
kovin in ponudbo storitev na novi žični eroziji.

Leta 2015 pa se je podjetje reorganiziralo po standardu ISO 9001.

Leta 2021 je podjetje dobilo optični merilni stroj ViciVision za boljše spremljanje
kvalitete proizvodnje.

\subsubsection{Cilji}
Podjetje se kljub svojemu omejeneu prostoru stalno razvija in izboljšuje
svoje postopke. Trenutno ima na voljo štiri robote za samodejno pregledovanje
in merjenje izdelkov z merilnimi uricami ter optično merjenje s kamerami.
V prihodnosti imajo cilj razširiti strojni park z več krivuljnimi avtomati
in CNC-stroji.
