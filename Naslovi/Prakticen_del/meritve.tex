\subsection{Merjenje natančnosti postopka}
Kos se izdeluje iz surovca - palice premera 5 mm z toleranco
H7 ki ima mejne vrednosti + 0 mm in - 0.012 mm, kar pomeni,
da premera mi ni potrebno posebej preverjati. Ker so palice
dobavljene v saržah, lahko izmerim samo nekaj palic preden
jih obdelam in lahko samo z certifikatom materiala, ki nam
ga izda dobavitelj, garantiram, da bo premer skladen z
zahtevami kupca.

Za zunanja posnetja ni nobenih predpisanih toleranc in tudi
kupec ne zahteva nič posebnega le da so robi posneti, zato
tudi posnetij mi ni potrebno posebej meriti. Glede notranjih
posnetij, pa je kupec malce bolj zahteven, zato se namreč
vsak kos pregleda z kamero, kjer se pregleda, če je izdelano
to posnetje in tudi če je narejena luknja.

Luknjo pa je potrebno približno dvakrat na uro preveriti z
kalibri premera 1.41 mm in 1.49 mm. Večji kaliber ne sme iti
čez luknjo, manjši pa more. Tolerančno območje je za 0.01 mm
manjše od prepisanega, kar si je podjetje samo izbralo zaradi
varnosti.

Dolžina 6.8 mm je pa najbolj kritična mera na izdelku.
Največ je odvisna od kvalitete in stanja obrabljenosti
odrezne ploščice. Zaradi tega je potrebno ob vsaki menjavi
palice, približno vsakih 10 min, preveriti dolžino z kalibrom
ali z pomičnim merilom. Spodaj na tabeli \ref{tabela_meritev}
so prikazani rezultati meritev dolžine 6.8 mm prvih 40 kosov,
katere sem izmeril z pomičnim merilom Mitutoyo ABSOLUTE,
ki ima merilno resolucijo 0.01 mm in imensko natančnost \(\pm\) 0.02 mm.

\begin{table}[H]
	\caption{Rezultati meritev pri zagonu stroja}
	\label{tabela_meritev}
	\begin{center}
		\begin{tabular}{|c|c|c|c|}
			\hline
			Izmerjena vrednost [mm] & Število \\
			\hline
			6.77                    & 3       \\
			\hline
			6.78                    & 5       \\
			\hline
			6.79                    & 10      \\
			\hline
			6.80                    & 16      \\
			\hline
			6.81                    & 12      \\
			\hline
		\end{tabular}
	\end{center}
\end{table}

Večina kosov je bila krajša od imenske mere ampak znotraj tolerančnega
polja. Če upoštevamo še merilno natančnost pomičnega merila \(\pm\) 0.02 mm,
zgornja in spodnja mejna vrednost ne presegata tolerančnega polja.
Kljub temu se lahko pri delovanju stroja zgodijo napake
in se po izdelavi, kosi še dolžinsko pregledajo na posebej
narejenem stroju za serijsko pregledovanje.