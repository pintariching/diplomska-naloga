\subsection{Izdelava krivulje}
Za operacijo odreza sem posebej izdelal krivuljo. Kot vzpona
sem zaokrožil na 150° in višino vzpona sem povečal na 15 mm,
kar pomeni, da sem moral uporabiti vzvod z razmerjem 1:5 na
sledilcu krivulje. Krivuljo sem izrisal in jo prenesel na
okrogel surovec iz jekla C45 s premerom 120 mm in debeline 8 mm, ki je imel
v sredini že izvrtano in postruženo luknjo s premerom 40 mm.
Potem sem na rezkalnem stroju s svedrom izvrtal luknje po obodu tako,
da se čim manj pokrivajo. Nekaterih lukenj mi ni uspelo natančno
izvrtati, zaradi česar se je po koncu vrtanja surovec še vedno držal skupaj
s krivuljo, ki sem jo izbil s kladivom. Na Sliki \ref{izdelava_krivulje}
je prikazan ostanek surovca po vrtanju, ko sem izbil krivuljo.

\begin{figure}[H]
	\begin{center}
		\includegraphics[width=6cm]{izdelava_krivulje.jpg}
		\caption{Ostanek surovca od vrtanja lukenj za izdelavo krivulje
			\cite{lasten}}
		\label{izdelava_krivulje}
	\end{center}
\end{figure}

Po tem sem krivuljo z ročno kotno brusilko izbrusil do vrisane linije
in izrezal še zarezo širine približno 35 mm, da se lahko krivulja
kasneje natakne na krivuljno gred. Izdelava je bila tako skoraj zaključena.
Preostalo je še, da  krivuljo zakalim v kalilni peči ter jo
še enkrat na fino pobrusim za čim bolj gladko površino,
da sledilec čim lepše potuje po njej.

Za kaljenje sem uporabil kalilno peč KAL 31, proizvajalca TerraArt,
prikazana je na Sliki \ref{kalilna_pec_slika}. Ta ima možnost programiranja
cikla segrevanja in nastavljanja hitrosti segrevanja in ohlajanja.

\begin{figure}[H]
	\begin{center}
		\includegraphics[width=8cm]{kalilna-pec-kal-31.jpg}
		\caption{Kalilna peč KAL 31
			\cite{kalilna_pec}}
		\label{kalilna_pec_slika}
	\end{center}
\end{figure}

Krivuljo sem po kaljenju zbrusil z ročno kotno brusilko in s tem
zaključil izdelavo. Preostala je še montaža na stroj in preizkus
delovanja. Če bi bilo potrebno kar koli popraviti, se lahko z
varilnim strojem navari tanjši sloj materiala in se ponovno zbrusi
površino. Če je to večje popravilo, potem je dobro krivuljo
žariti, da popustijo napetosti od varjenja, in ponovno zakaliti.
Če tega ne storimo, se bo po popravilu krivulja veliko hitrejše
obrabila in obstaja tudi možnost, da se navarjeni del odlomi, se zatakne ob krivuljno gred in ohišje ter pripelje do večjega
strojeloma.
