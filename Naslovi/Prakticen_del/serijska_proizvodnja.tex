\subsection{Serijska proizvodnja}
Če so vse mere urejene, se lahko začne serijska proizvodnja.
Ko vstavimo novo palico, skoraj vedno prvi izdelan kos ni dober,
ker imajo palice na obeh koncih pobran rob in neravno čelno ploskev.
To bi se dalo rešit, če bi v ciklu izveli še operacijo čeljenja,
ampak pri velikih količinah, se bolj splača prvi kos zavrečt,
kot pa dodajanje nove operacije. Ko stroj obdela celotno palico,
ima podajalec stikalo, ki se sproži, ko povleče suport zadnji
kos. To nenudoma vstavi celoten stroj in prižge signalno luč,
ki nam da vizualno signalizacijo, da je stroj zaključil z delovanjem.
Za tem mora delavec odstraniti ostanek palice, ki ostane v vretenu
in ročno v podajalec namestiti novo palico in jo počasi pripeljati
do odreznega noža, ki služi kot distančnik za prvi kos. Za tem je
potrebno stroj le še prižgati in vklopiti sklopko za krivuljno gred
in avtomat ponovno deluje.

Po obdelavi na stružnem avtomatu, gredo izdelki na čiščenje z centrifugo,
katera izloči večji del hladilnega olja, ki je še ostalo na izdelku.
To olje se nato ponovno iztoči nazaj v stroje, kjer se ponovno uporablja.
Po tem, se izdelki še sperejo z vročo vodo in čistilom in se posušijo
v peči. Po sušenju so kosi čisto brez sledi olja. To je zelo pomembno
za nadaljni proces pregledovanja, saj bi se roboti za pregledovanje
hitro umazali.