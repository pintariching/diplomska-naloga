\subsection{Serijska proizvodnja}
Če so vse mere urejene, se lahko začne serijska proizvodnja.
Ko vstavimo novo palico, prvi izdelan kos skoraj nikoli ni dober,
ker imajo palice na obeh koncih pobran rob in neravno čelno ploskev.
To bi se dalo rešiti, če bi v ciklu izveli še operacijo čeljenja, a se pri velikih količinah bolj splača zavreči prvi kos
kot pa dodajatu nove operacije. Ko stroj obdela celotno palico,
ima podajalec stikalo, ki se sproži, ko povleče suport zadnji
kos. To nenudoma ustavi celoten stroj in prižge signalno luč,
ki nam da vizualno signalizacijo, da je stroj zaključil z delovanjem.
Zatem mora delavec odstraniti ostanek palice, ki ostane v vretenu
in ročno v podajalec namestiti novo palico ter jo počasi pripeljati
do odreznega noža, ki služi kot distančnik za prvi kos. Nato je
potrebno stroj le še prižgati in vklopiti sklopko za krivuljno gred, nakar avtomat ponovno deluje.

Po obdelavi na stružnem avtomatu gredo izdelki na čiščenje s centrifugo,
ki izloči večji del hladilnega olja, ki je še ostalo na izdelku.
Le-to se nato iztoči nazaj v stroje, kjer se ponovno uporablja.
Po tem se izdelki še sperejo z vročo vodo in čistilom ter se posušijo
v peči. Po sušenju so kosi čisto brez sledi olja. To je zelo pomembno
za nadaljni proces pregledovanja, saj bi se roboti za pregledovanje
hitro umazali.