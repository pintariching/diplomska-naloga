\subsection{Montaža orodja}
Orodje za odrez in pobiranje roba se vpne v posebne držače
z pomočjo vijakov, ki pritiskajo na steblo noža. Postopek je ponavadi tak,
da na roke obračamo krivuljno gred, dokler vodilo ne pride
na začetek vzpona krivulje, kjer se začne odrezavanje. Takrat vpnemo orodje
in ga oddaljimo oddaljeno za približno 0.5-1 mm od površine
materjala. Tako naredimo za vsa orodja, kot tudi za svedre,
katere pa vpnemo v stročnice.
Stroj nam tudi omogoča fine nastavitve orodij po obeh osih
in nam ponuja možnost dokaj natančnih izdelav. Z novo
krivuljo, orodjem in pravilnimi parametri, lahko dosegamo
natančnosti do ±0.01 mm.
