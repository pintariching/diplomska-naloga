\subsection{Montaža orodja}
Orodje za odrez in pobiranje roba se vpne v posebne držače
s pomočjo vijakov, ki pritiskajo na steblo noža. Postopek je ponavadi tak,
da na roke obračamo krivuljno gred, dokler sledilec ne pride
na začetek vzpona krivulje, kjer se začne odrezavanje. Takrat vpnemo orodje
in ga rahlo potisnemo na površino materiala. Nato krivuljno gred
obrnemo čisto malo v nasprotno smer, da orodje ne drgne po materialu
in vklopimo vrtljaje ter preizkusimo, kako nož odrezuje. Po ostružkih
in zvoku lahko ocenimo, če je orodje pravilno brušeno in če je vpeto
na pravilni višini glede na središče vrtenja. Če je orodje višje od
središča, se lahko zgodi, da s prosto površino drgne ob material in
se prekomerno segreva, lahko se tudi zgodi, da sploh ne struži.
Manjši problem je, če je orodje prenizko, saj lahko še zmeraj
struži. Tako naredimo za vsa orodja kot tudi za svedre,
ki pa jih vpnemo v stročnice v nasprotno vreteno. Pri preizkušanju
orodja moramo paziti, da nimamo premajhne rezalne hitrosti, če
vrtimo na roke, saj lahko ob struženju nerjavečega jekla orodje
hitro prekomerno segrejemo in ga uničimo. Pri avtomatnemu jeklu
je to manjši problem, saj je veliko mehkejše in lažje za obdelavo.
Gauthier GM127 nam tudi omogoča fine nastavitve orodij po obeh oseh
preko vijakov s finim navojem, ki nam omogočajo dokaj natančno
nastavitev orodja. Z novo krivuljo, orodjem in pravilnimi parametri
lahko dosegamo natančnosti tudo do ±0.01 mm.
