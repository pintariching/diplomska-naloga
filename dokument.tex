\documentclass[a4paper, 12pt]{article}

\usepackage[utf8]{inputenc}
\usepackage{times}

% Robovi
\usepackage[a4paper, left=3cm, right=2.5cm, top=3cm, bottom=3cm, footskip=1.5cm]{geometry}

\usepackage[slovene]{babel}
\usepackage{titling}
\usepackage{authblk}
\usepackage{lipsum}
\usepackage[table]{xcolor}
\usepackage{multirow}
\usepackage{float}
\usepackage{amsmath}
\usepackage{parskip}
\usepackage{gensymb}
\usepackage{subfiles}
\usepackage{datetime}
\usepackage{fancyhdr}
\usepackage{titlesec}

% Nastavitev formata datumov
\newdateformat{monthyeardate}{
    \monthname[\THEMONTH] \THEYEAR}
\newdateformat{yeardate}{
    \THEYEAR}

% Nastavitev barve linkov
\usepackage[colorlinks, unicode]{hyperref}
\hypersetup{
    colorlinks=true,
    linkcolor=black,
    filecolor=black,
    citecolor=black,
    urlcolor=blue,
    hypertexnames=true,
}

% Nastavitev velikosti pisav naslovov
\titleformat*{\section}{\fontsize{14}{18}\selectfont\bfseries\MakeUppercase}
\titleformat*{\subsection}{\fontsize{12}{12}\selectfont\bfseries\MakeUppercase}
\titleformat*{\subsubsection}{\fontsize{12}{12}\selectfont\bfseries}

% Ne upoštevaj kazala, kazala slik in kazala tabel v kazalu
\usepackage[nottoc,notlot,notlof]{tocbibind}

% Nastavitev razmika med stolpci tabel
\usepackage{multicol}
\setlength{\columnsep}{2.5cm}

% Nastavitev poti do mape z slikami
\usepackage{graphicx}
\graphicspath{ {./Slike/} }

% Prevod naslova kazala
\addto\captionsenglish{
  \renewcommand{\contentsname}
    {Kazalo}
}

\renewcommand{\baselinestretch}{1.5}

% Nastavitev naslova, avtorja in datuma
\title{Nastavljanje stružnega avtomata}
\author{Tilen Viljem Pintarič}
\date{\today}

%
% Začetek dokumenta
%
\begin{document}
\subfile{Preduvod/prva_stran.tex}

% Prazna stran brez noge in stilov
\newpage\null\thispagestyle{empty}\newpage

\subfile{Preduvod/naslovnica.tex}

% Nastavi številko strani v nogi
\pagestyle{fancy}
\fancyhf{}
\renewcommand{\headrulewidth}{0pt}
\fancyfoot[R]{\fontsize{12}{18} \thepage}

\pagenumbering{Roman}
\subfile{Clanki/zahvale_in_povzetki.tex}
\newpage

\tableofcontents
\newpage

\listoffigures
\newpage

\listoftables
\newpage

\pagenumbering{arabic}
\section{Uvod}
\subsection{Predstavitev podjetja}
\subfile{Clanki/predstavitev_podjetja.tex}

\section{Teoretičen del}
\subsection{Stružnica}
\subfile{Clanki/struznica.tex}
\newpage

\subsection{Avtomatska stružnica}
\subfile{Clanki/avtomatska_struznica.tex}
\newpage

\subsection{Eno-vretenski avtomat}
\subfile{Clanki/eno-vretenski_avtomat.tex}

\subsection{Več-vretenski avtomat}
\subfile{Clanki/vec-vretenski_avtomat.tex}

\subsection{Izdelava krivulj}
\subfile{Clanki/izdelava_krivulj.tex}

\subsection{O orodjih}
\subfile{Clanki/o_orodjih.tex}

\newpage
\section{Praktičen del}
\subfile{Clanki/predstavitev_kosa.tex}
\subfile{Clanki/postopek_izdelave.tex}
\subfile{Clanki/izracun_krivulj.tex}
\subfile{Clanki/izbira_jerjemenov.tex}
\subfile{Clanki/montaza_strocnic.tex}
\subfile{Clanki/montaza_orodja.tex}
\subfile{Clanki/serijska_proizvodnja.tex}

\subfile{Clanki/predlog_za_prakso.tex}

\subfile{Clanki/zaklucek.tex}


\newpage
\bibliographystyle{plain}
\renewcommand{\refname}{Viri}
\bibliography{Viri/viri.bib}

\subfile{Clanki/priloge.tex}
\end{document}